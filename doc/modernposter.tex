% modernposter.tex
%
% This file is part of the modernposter LaTeX template
%
% Version 1.02.1 2018/01/04
%
% Copyright 2018 David Derler
%
% This work is licensed under the Creative Commons Attribution-ShareAlike 4.0 
% International license (cf. https://creativecommons.org/licenses/by-sa/4.0/)  
\documentclass{article}

\usepackage{xcolor}
\usepackage[hidelinks=true]{hyperref}
\usepackage{listings}
\usepackage{examplep}
\usepackage{amsmath}
\lstset{
  basicstyle=\footnotesize\ttfamily,
  breaklines=true,
  language=TeX,
  numbers=left,
  numbersep=5pt,
  numberstyle=\tiny\color{lightgray},
  stepnumber=2,
  tabsize=2,
  backgroundcolor=\color{black!4}
}

\title{modernposter - A modern \LaTeX~poster theme}
\author{Maintained by David Derler\thanks{List of contributors: \url{https://github.com/derlerd/modernposter/graphs/contributors}}}
\date{Version 1.02.1 2018/01/04}

\begin{document}
  \maketitle

  \begin{abstract}
\noindent This class extends the a0poster class in that it adds support to easily create posters without the need for taking care of the layout at all. It allows to use {\tt $\backslash$maketitle} to generate a fancy header containing the title information and also provides macros to position various different types of text boxes in a two-column layout. The color scheme  is inspired by the metropolis beamer theme.
  \end{abstract}
  \section{Installation}
  The package can be installed by copying the {\tt modernposter.cls} file to the folder of your \LaTeX~distribution, which contains the manually installed packages.\footnote{For TeXLive distributions this should be {\tt $\sim$/texmf/tex/latex/modernposter/}.} Alternatively, you may choose to clone the github repository\footnote{\url{https://github.com/derlerd/modernposter}} containing the development version of this package and use the {\tt install} target of the Makefile contained in this repository. This target will automatically install the package at {\tt $\sim$/texmf/tex/latex/modern\-poster/} and the documentation at {\tt $\sim$/texmf/doc/latex/modernposter/}.
  \section{Usage}
  This package is intended to save time whenever a poster needs to be created. Thus we have tried to keep the package as easy to use as possible. Below, we describe the available options as well as the available macros. In addition, we provide a toy example which should illustrate the usage of the package. For more details also refer to the {\tt demo.tex} file in the package.
  \subsection{Options}
  The following options can be set. Some option keys allow to optionally supply an explicit value.  
  \begin{description}
  	\item[\PVerb{logo=[logo]}] This option allows to provide a path to a logo which is then included on the top right of the page. For example, {\tt logo=img.png}.
  	\item[\PVerb{hlcolor=[color]}] This option allows to override the highlight color. The color needs to be supplied in HTML format. For example, {\tt hlcolor=F70146}. 
    \item[\PVerb{helvet}] This option, if supplied, switches the font from Fira Sans to Helvetica.
  \end{description}
  \subsection{Environments}
  The following environments are provided by the package.
  \begin{description}
  	\item[\tt postercolumn:] The template is designed for a two-column layout. This environment represents a column. 
  \end{description}
  \subsection{Commands}
  The following commands are provided by the package.
  \begin{description}
  	\item[\PVerb{\posterbox{[t]}{[c]}}:] This command can be used within a {\tt postercolumn} environment to place a text box within a column. It takes two arguments being the title \PVerb{t} as well as the content \PVerb{c} of the box.
  	\item[\PVerb{\doubleposterbox[height]{[t1]}{[c1]{[t2]}{[c2]}}}:] This command can be used within a {\tt postercolumn} environment to place two text boxes. The combined height of those two text boxes is identical to the height of a text box placed with {\tt posterbox}. There is an optional parameter {\tt height}, which allows to define the height of the first text box relative to the height of a text box placed with {\tt posterbox}. It takes four arguments being the title {\tt t1} and the content {\tt c1} of the first text box and the title {\tt c2} and the content {\tt c2} of the second text box, respectively.
  \end{description}
  \subsection{Toy Example}
  The following listing demonstrates the basic usage of the package.
  \begin{lstlisting}
\documentclass{modernposter}

\title{This is a Modernposter Demo}
\author{Firstname Lastname}
\email{firstname.lastname@domain.com} 

\begin{document}
  \maketitle  

  \begin{postercolumn}
    \posterbox{Introduction}{ 
      $\dots$
    }
  \end{postercolumn} 
  \begin{postercolumn}
    \doubleposterbox[0.6]
      {Problem Statement}{
        $\dots$
      }
      {\highlight{Results}}{ 
        $\dots$
      }
  \end{postercolumn}
\end{document}

  \end{lstlisting}

  \section{Getting Involved}
  We are pretty sure that there is quite some potential for improvements. Please send feedback to \href{mailto:modernposter@derler.info}{\tt modernposter@derler.info} or consider getting involved - the development repository is hosted on GitHub and can be found at \url{https://github.com/derlerd/modernposter}. 

  \section{License}
  The code is licensed under the \href{https://creativecommons.org/licenses/by-sa/4.0/}{Creative Commons Attribution-ShareAlike 4.0 International license}.

  \section{Version History}
  \begin{description}
  	\item[1.0] Initial release.
  	\item[1.01.2] Improved versioning; Added explicit version numbers to every file.
  	\item[1.02.1] Implemented option to switch between Fira Sans and Helvetica; Removed some unused stuff \& reduced dependencies on other packages; Improved documentation
  \end{description}
\end{document}